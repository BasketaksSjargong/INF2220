\documentclass[paper=a4, fontsize=11pt]{scrartcl}
\usepackage[T1]{fontenc} % Use 8-bit encoding that has 256 glyphs
\usepackage{fourier} % Use the Adobe Utopia font for the document - comment this line to return to the LaTeX default
\usepackage[english]{babel} % English language/hyphenation
\usepackage{amsmath,amsfonts,amsthm} % Math packages
\usepackage{graphicx} % Embedding graphics in document
\usepackage{minted}
\usepackage{hyperref}
\usepackage{caption}
\usepackage{listings}
% Custom code listing
\usepackage{color}
\definecolor{codegreen}{rgb}{0,0.6,0}
\definecolor{codegray}{rgb}{0.5,0.5,0.5}
\definecolor{codepurple}{rgb}{0.58,0,0.82}
\definecolor{backcolour}{rgb}{0.95,0.95,0.92}
\lstdefinestyle{mystyle}{
  backgroundcolor=\color{backcolour},
  commentstyle=\color{codegreen},
  keywordstyle=\color{magenta},
  numberstyle=\tiny\color{codegray},
  stringstyle=\color{codepurple},
  basicstyle=\footnotesize,
  breakatwhitespace=false,
  breaklines=true,
  captionpos=b,
  keepspaces=true,
  numbers=left,
  numbersep=5pt,
  showspaces=false,
  showstringspaces=false,
  showtabs=false,
  tabsize=2
}
\lstset{style=mystyle}
                                % End of custom code listing
\usepackage[utf8]{inputenc}
\usepackage{lipsum} % Used for inserting dummy 'Lorem ipsum' text into the template
\usepackage{sectsty} % Allows customizing section commands
\allsectionsfont{\centering \normalfont\scshape} % Make all sections centered, the default font and small caps
\usepackage{fancyhdr} % Custom headers and footers
\pagestyle{fancyplain} % Makes all pages in the document conform to the custom headers and footers
\fancyhead{} % No page header - if you want one, create it in the same way as the footers below
\fancyfoot[L]{} % Empty left footer
\fancyfoot[C]{} % Empty center footer
\fancyfoot[R]{\thepage} % Page numbering for right footer
\renewcommand{\headrulewidth}{0pt} % Remove header underlines
\renewcommand{\footrulewidth}{0pt} % Remove footer underlines
\setlength{\headheight}{13.6pt} % Customize the height of the header
\numberwithin{equation}{section} % Number equations within sections (i.e. 1.1, 1.2, 2.1, 2.2 instead of 1, 2, 3, 4)
\numberwithin{figure}{section} % Number figures within sections (i.e. 1.1, 1.2, 2.1, 2.2 instead of 1, 2, 3, 4)
\numberwithin{table}{section} % Number tables within sections (i.e. 1.1, 1.2, 2.1, 2.2 instead of 1, 2, 3, 4)
\setlength\parindent{0pt} % Removes all indentation from paragraphs - comment this line for an assignment with lots of text
\theoremstyle{definition}
\newtheorem*{definition}{Definition}
                                % Theorems
                                %----------------------------------------------------------------------------------------
                                % TITLE SECTION
                                %----------------------------------------------------------------------------------------
\newcommand{\horrule}[1]{\rule{\linewidth}{#1}} % Create horizontal rule command with 1 argument of height
\title{ 
  \normalfont \normalsize
  \textsc{University of Oslo} \\ [25pt] % Your university, school and/or department name(s)
  \horrule{0.5pt} \\[0.4cm] % Thin top horizontal rule
  \huge INF2220 - Summary \\ % The assignment title
  \horrule{2pt} \\[0.5cm] % Thick bottom horizontal rule
}
\author{Ivar Haugaløkken Stangeby} % Your name
\date{\normalsize\today} % Today's date or a custom date
\begin{document}
\maketitle % Print the title
\tableofcontents
\section{Graphs}

\subsection{Definitions}
\begin{definition}[Connected graph]
  An undirected graph is called \textbf{connected} if there is a path from any vertex to any other vertex in the graph. If the graph is directed, we call it \textbf{strongly connected}.
\end{definition}
\begin{definition}[Biconnected graph]
  A \textbf{biconnected} graph is a connected graph with the property that if any vertex were to be removed, the graph will remain connected.   
\end{definition}

\begin{definition}[Minimum spanning tree]
  Given a connected, undirected graph, a spanning tree of that graph is a subgraph that is a tree and connects all the vertices together. We can also assign a weight to each edge and use it to assign a weight to a spanning tree. A \textbf{minimum spanning tree} (MST) is a spanning tree with weight less than or equal to the weight of every other spanning tree.
\end{definition}

\begin{definition}[Acyclic graph]
  An \textbf{acyclic graph} is a directed graph that contains no cycles
\end{definition}

\begin{definition}[Topological sort] 
  A \textbf{topological sort} is an ordering of 
  vertices in a directed acyclic
  graph, such that if there is a path from \(v_i\) to \(v_j\),
  then \(v_j\) appears \textit{after} \(v_i\) in the ordering.  
\end{definition}

\begin{definition}[Indegree]
  The \textbf{indegree} of a vertex \(v\) is defined as the number of edges \(\left(u,v\right)\).
\end{definition}

\subsection{Representation of graphs}

\paragraph{Adjacency matrix} There are several ways to represent a graph. The
most simple way is called an \textbf{adjacency matrix}. The adjacency matrix is
a two-dimensional array where for each edge \( \left( u, v \right)\) we set
\(\mathrm{A}[u][v]\) to \textbf{true}.  This representations is good for dense
graphs. This means that there are a lot of edges in relation to the number of
vertices.The space requirement is \(\Theta\left( |V|^2 \right)\)

\paragraph{Adjacency List}
If the graph on the other hand is not dense, an \textbf{adjacency list} representation is a better option. For each vertex, keep a list of the adjacent vertices. The space requirement is then \( O \left( |E| + |V| \right)\).

\subsection{Graph algorithms}

\subsubsection{Topological Sort}
\hyperref[toposort]{\textbf{See Pseudocode}}
It is clear that any cyclic graphs can not be topologically sorted.
The ordering is not necessarily unique, thus you might get different results.
The usual algorithms for topological sorting have a running time
\( O \left( |V| + |E| \right) \).

\begin{enumerate}
  \item Find a vertex with no incoming edges (indegree zero)
  \item Print this vertex, and remove it, and its edges, from the graph.
  \item Repeat from step 1 until all vertices has been checked.
\end{enumerate}
\begin{figure}[p]
  \begin{minted}[bgcolor=backcolour]{java}
    void toposort() throws CycleFoundException{
      Queue<Vertex> q = new Queue<Vertex>();
      int counter;

      for each vertex v
      if (v.indegree = = 0)
      q.enqueue(v);

      while (!q.isempty()) {
        Vertex v = q.dequeue()
        v.topNum = ++counter;

        for each Vertex w adjacent to v
        if ( --w.indegree = = 0)
        q.enqueue(w);
      }
      if (counter != NUM_VERTICES )
      throw new CycleFoundException();
    }
  \end{minted}
  \caption{Pseudocode to perform topological sort}
  \label{toposort}
\end{figure}

\subsubsection{Dijkstra's algorithm}
\hyperref[dijkalg]{\textbf{See Pseudocode}}
Dijkstra's algorithm is a graph search algorithm that solves the single-source
shortest path problem for a graph with non-negative edge path costs.  For a
given source vertex in the graph, the algorithm finds the path with the lowest
cost between that vertex, and any other vertex.  In essence, the algorithm
picks the unvisted vertex with the lowest distance, calculates the distance
through it to each unvisited neighbor and updates the neighbor's distance if it
is smaller. Mark the vertex as visited when all the neighbors have been
checked.

It has a worst case performance of \(O\left(|E| + |V|\log|V|\right)\)

\begin{enumerate}
  \item Set starting point distance to 0, and \(\infty\) for the rest.
  \item Mark all nodes as unvisited, set initial node to current-node. Fill a list with all the unvisited nodes.
  \item For current node - consider all of its unvisited neighbors and calculate their tentative-distance = currentnodedistance + cost of edge traversal.
  \item When done considering all the neighbors of the current, mark current as visited and remove it from unvisited-list.
  \item If the destination node has been marked visited, or if the smallest tentative distance among the nodes in unvisited set is \(\infty\), then stop. The algorithm has finished.
  \item Select the unvisited node with the smallest tentative distance, and set it as new current-node and repeat from step 3.
\end{enumerate}

\begin{figure}[p]
  \begin{minted}[bgcolor=backcolour]{java}
    void dijkstra(Vertex s){
      for each Vertex v{
        v.dist = infinity;
        v.known = false;
      }
      
      s.dist = 0;

      while(there is an unknown distance vertex){
        Vertex v = smallest unknown distance vertex;
        v.known = true;
        
        for each Vertex w adjacent to v{
          if (!w.known){
            DistType cvw = cost of edge from v to w;

            if (v.dist + cvw < w.dist){
              decrease(w.dist to v.dist + cww);
              w.path = v;
            }
          }
        }
      }
    }
  \end{minted}
  \caption{Pseudocode for Dijkstra's algorithm}
  \label{dijkalg}
\end{figure}
\subsubsection{Prim's algorithm}
Prim's algorithm is a greedy algorithm that finds a minimum spanning tree for a
connected weighted undirected graph.

Its complexity relies entirely on the data structure used. Using an adjacency matrix the algorithm has a worst case performance of \(O\left(|V|^2\right)\).
\begin{enumerate}
  \item Initialize a tree with a single vertex, chosen arbitrarily from the graph
  \item Find the minimum weight edge from a vertex in the tree to a vertex not in the tree and transfer it to the tree.
  \item Repeat step 2 - until all vertices are in the tree.
\end{enumerate}

\subsubsection{Kruskal's algorithm}
Kruskal's algorithm is a greedy algorithm that finds a minimum spanning tree
for a connected weighted graph.  If the graph is not connected, it finds a
minimum spanning forest.

\begin{enumerate}
  \item Create a forest \(F\), where each vertex in the graph is a separate tree.
  \item Create a set \(S\), containing all the edges in the graph.
  \item while \(S\) is nonempty and \(F\) is not yet spanning.
    \begin{enumerate}
      \item remove an edge with minimum weight from \(S\).
      \item if that edge connects two different trees, then add it to the forest, combining the two trees into a single tree.
    \end{enumerate}
\end{enumerate}

\end{document}
